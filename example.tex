%!TEX program = xelatex
% ==================== 文档类选项 ====================
% 学位类型: master(硕士) 或 doctor(博士)
% 论文版本: blind(盲审) / defense(答辩) / final(正式)
\documentclass[master,final]{zufe-thesis}

% ==================== 自定义字体设置(可选)====================
% 若需使用本地 TTF 字体,请取消注释并指定路径
% \setCJKmainfont[Path=./]{SimSun.ttf}
% \setCJKsansfont[Path=./]{SimHei.ttf}
% \setCJKmonofont[Path=./]{KaiTi.ttf}
% \setmainfont[Path=./]{TimesNewRoman.ttf}

% ==================== 额外宏包 ====================
\usepackage{unicode-math} %在此加载你需要的宏包
\usepackage{lipsum}

% ==================== 参考文献 ====================
\addbibresource{example.bib} % 请确保 example.bib 存在或替换为你的 .bib 文件

% ==================== 论文基本信息(占位符)====================
\thesisTitle{基于动态优化模型的劳动力迁移行为研究}
\thesisTitleEN{A Study on Labor Migration Behavior Based on Dynamic Optimization Models}
\authorName{张三}                % 示例姓名
\studentID{202300000000}         % 示例学号
\mentorName{李四}                % 示例导师
\majorName{理论经济学}
\deptName{经济学院}
\submitDate{2025年6月}
\submitDateEN{June 2025}
\reviewDate{2025年6月} % 盲审日期
\secrecyLevel{公开} % 论文密级

% ==================== 正文开始 ====================
\begin{document}

% -------------------- 封面 --------------------
\makethesiscover

% -------------------- 声明页(盲审版自动省略)--------------------
\makestatement

% -------------------- 中英文扉页 --------------------
\makechinesetitlepage
\makeenglishtitlepage

% -------------------- 前置部分(罗马页码)--------------------
\frontmatter

% 中文摘要
\begin{abstract}
\lipsum[1-2]

\keywords{劳动力迁移;动态优化;回流行为;居住地选择}
\end{abstract}

% 英文摘要(关键词自动转小写)
\begin{abstracten}
\lipsum[1-2]

\keywordsen{labor migration; dynamic optimization; return behavior; residential location choice}
\end{abstracten}

% 目录
\tableofcontents

% -------------------- 正文部分(阿拉伯页码)--------------------
\mainmatter

\chapter{绪论}
\section{研究背景与意义}
\lipsum[3]

\subsection{文献综述}
国外学者如 \textcite{greenwood1997} 早期研究……。

\section{研究框架}
\lipsum[4]

\chapter{理论模型}
\section{模型设定}
考虑代表性个体在时期 \( t \) 的效用函数:
\begin{equation}\label{eq:utility}
U_t = \ln(C_t) + \theta \cdot V(L_t) - \phi \cdot D(L_t, L_{t-1})
\end{equation}
其中 \( C_t \) 为消费,\( L_t \) 为居住地,\( D(\cdot) \) 为迁移成本。

\begin{table}[htbp]
\centering
\caption{主要变量描述性统计}
\label{tab:stats}
\begin{tabular}{lcc}
\toprule
变量 & 均值 & 标准差 \\
\midrule
是否迁移 & 0.18 & 0.38 \\
月收入(千元) & 6.2 & 4.1 \\
教育年限 & 10.3 & 3.7 \\
\bottomrule
\end{tabular}
\end{table}


% -------------------- 后置部分 --------------------
\backmatter

% 参考文献
\printbibliography[heading=bibintoc, title=参考文献]

% 附录
\appendix
\chapter{模型推导细节}
附录公式编号为式~\ref{eq:appendix}。
\begin{equation}\label{eq:appendix}
\frac{\partial U}{\partial L} = 0
\end{equation}

\chapter{补充图表}
附录表格编号为表~\ref{tab:appendix}。
\begin{table}[htbp]
\centering
\caption{稳健性检验结果}
\label{tab:appendix}
\begin{tabular}{lc}
\toprule
变量 & 系数 \\
\midrule
年龄 & 0.041** \\
(0.018) & \\
\bottomrule
\end{tabular}
\end{table}

% 发表论文列表(盲审版自动省略)
\begin{publications}
\begin{enumerate}
\item 张三, 李四. 劳动力迁移的动态机制研究[J]. 经济研究, 2024(5): 100–115.
\item Zhang, S., Li, S. Dynamic Migration and Regional Inequality[J]. Journal of Development Economics, 2023, 162: 102–120.
\end{enumerate}

参与的科研项目:
\begin{enumerate}
\item 新型城镇化背景下人口流动研究. 国家社会科学基金项目. 2022–2025.
\end{enumerate}
\end{publications}

% 致谢(盲审版自动省略)
\begin{acknowledgement}
本研究得以顺利完成,离不开导师的悉心指导、家人的无私支持以及同行的宝贵建议。特别感谢浙江财经大学提供的学术平台与资源支持。文中不足之处,敬请批评指正。
\end{acknowledgement}

% 封底
\makebackcover

\end{document}